\documentclass[a4paper]{article}

\begin{document}

\begin{titlepage}
    \title{Solitario di Prina}
    \author{Gallina Roberto}
    \date{13/11/2022}
    \maketitle
\end{titlepage}

\newpage

\tableofcontents

\newpage

\section{Premesse}
In tutto il documento ci si riferirà a il \emph{mazzo (Deck)}, esso è da intendere come un mazzo di 40 carte, divise in 4 semi; per convenzione durante il documento si userà il mazzo francese, in cui le carte sono l'uno (1), il due (2), il tre (3), il quattro (4), il cinque (5), il sei(6), il sette (7), il fante (J), la regina (Q) e il re (K); mentre i semi sono cuori (), quadri (), fiori () e picche().

% TODO: Add seme

% Immagine delle carte

\newpage

\section{Funzionamento del gioco}

Il gioco è molto semplice, dato un mazzo da gioco mischiato, si tengono tutte le carte coperte in pila; si scopre le prime tre carte.
Nel caso la prima e la terza carta hanno stesso valore o stesso seme, allora la seconda carta viene spostasta sopra la prima (avvicinando la terza).
Successivamente si scopre un'altra carta di ricomincia a controllare dall'inizio.

% Add image del funzionamento

\subsection{Sequenza corretta}

Viene definita \emph{sequenza corretta}, una sequenza, in cui terminate le carte si hanno due esattamente due pile di carte.


% Esempio 


\subsection{Sequenza perfetta}

Viene definita \emph{sequenza perfetta}, una sequenza corretta, in cui è esattamente l'ultima carta a formare la seconda pila; per cui terminate le carte di hanno esattamente due pile di carte in cui la seconda pila ha esattamente una carta


% Esempio 


\subsection{Sequenza n-perfect}

Viene definita \emph{sequenza n-perfect}, una sequenza, in cui terminate le carte non ci sono pile con più di un carta, ossia il numero delle pile è uguale al numero delle carte

% Esempio


\newpage

\section{Statistiche}



\section{Premesse}

CIAO






\end{document}